% !TeX root = EIRSAT_ADCS_Model.tex

\section{Introduction}
This document details initial modelling and simulation of the Attitide determination and Control System (ADCS) for EIRSAT-1.
This includes a background theory section, details of the simulation environment and a number of simulation results representing different control modes and different mission profiles.

\section{ADCS System Specification}

\subsection{Control Modes}
\begin{itemize}
\item Detumble
\item Nadir pointing
\item Sun pointing
\item Spin?
\item Safe mode
\end{itemize}

\subsection{ACDS Requirements}

Max rate to detumble?

Required pointing accuracy?

Settling time?

\subsection{Failure modes}

Loss of sensor

Loss of actuator

\section{Modelling}
\subsection{Co-ordinate systems and Reference Frames}

The spacecraft moves in an inertial reference frame $N$ with associated cartesian coordinate system $OXYZ$ where $\mathbf{n_1}$, $\mathbf{n_2}$ and $\mathbf{n_3}$ are unit vectors along the $X$, $Y$ and $Z$ axes respectively.

A second reference frame $A$ is attached to the spacecraft rigid body with coordinate system $Cxyz$ where $C$ is the mass centre of the rigid body and $\mathbf{a_1}$, $\mathbf{a_2}$ and $\mathbf{a_3}$ are unit vectors along the $x$, $y$ and $z$ axes respectively.

A third reference frame ........  orbital frame

\subsection{Satellite Dynamics}

EIRSAT-1 is modelled as a single rigid body with 6-DOF (degrees of freedom).
The satellite has mass $m$ and inertia tensor $I$ in the body fixed coordinate system

\subsection{Rotational Kinematics}

The attitude of the spacecraft is described by quaternions.

\subsubsection{Alternative representations of attitude and conversions}
The orientation of the spacecraft may also be described using a Direction Cosine Matrix.

The orientation of the spacecraft may also be described using Euler angles.
Starting with the body-fixed $oxyz$ axes aligned with the $OXYZ$ axes, the body undergoes a sequence of rotations by angles $\theta_3$, $\theta_2$ and $\theta_1$ about the body-fixed $z$ (yaw), $y$ (pitch) and $x$ (roll) axes respectively to reach its final orientation.

The DCM (Direction Cosine Matrix) is then described by equation \ref{eq:dcm}.
\begin{equation} \label{eq:dcm}
C = 
\end{equation}

\subsection{Orbital Dynamics}

\subsection{Environment}
\subsubsection{Magnetic Field}
\subsubsection{Gravitational Field}

\subsection{Sensor Models}

\subsection{Actuator Models}

\section{Controller Design}
\subsection{Dynamics block}
\subsection{Environment block}
\subsection{Sensor block}

\section{Simulation Environment}
Simulink Model Structure

\subsection{Dynamics block}
\subsection{Environment block}
\subsection{Sensor block}
\subsection{Actuator block}
\subsection{Controller block}
\subsection{Reference block}
\subsection{Mode Select block}

\section{Running Simulations}
\subsection{Model Setup File}
\subsection{Simulation Setup File}
