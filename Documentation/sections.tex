% !TeX root = EIRSAT_ADCS_Model.tex

\section{Introduction}
This document details a study of the modelling and simulation of the Attitide Determination and Control System (ADCS) for the EIRSAT-1 cubesat.
This includes the satellite attitude dynamics, orbit propagation after release from the ISS; disturbances torques, position and attitude sensor models, magnetic actuator models and attitude feedback control system.
The simulation environment was developed using MATLAB/Simulink.
Simulation results are presented for several different test cases.
These test cases include demonstrations of different control modes, different control systems and different actuator and sensor configurations including failure of one or more devices.

\section{Objectives}
\begin{itemize} 
\item Create a simulation environment for the EIRSAT-1 ADCS subsystem including:
\begin{itemize} 
\item Satellite attitude dynamics, modelled as a single rigid body with 6DOF
\item Orbit propagation of the satellite after release from the ISS
\item Disturbance torques including: gravity gradient, solar radiation pressure, aerodynamic, residual magnetic field
\item orbit propagation after release from ISS
\end{itemize}
\item Define the different ADCS operational modes and functional requirements for each mode
\item Design feedback controllers for each mode of the ADCS:
\begin{itemize} 
\item first assuming ideal 3-axis actuators
\item then with two magnetorquer actuators
\end{itemize}
\item Design algorithm for safely switching between control modes
\item Design controllers for failure modes including:
\begin{itemize} 
\item failure of one magnetorquer
\item failure of attitude sensors (needs to be more clearly defined)
\end{itemize}
\end{itemize} 

\section{ADCS Modes and Functional Requirements}

\begin{itemize}
\item Detumbling
\item Sun/object pointing
\item Nadir/zenith pointing
\end{itemize}

\subsection{Detumbling} %Joe
Max rate to detumble?
Time taken to detumble?
Power?

\subsection{Sun/distant object pointing} % Victor
Required pointing accuracy?
Settling time from one place to another?
Power?

\subsection{Nadir/zenith pointing} % Daire
Required pointing accuracy?
Time?
Power?

\subsection{Switching between modes} % Joe
Which switches can happen?
When can they happen ? (where in orbit, in eclipse?)
Time taken?

\section{Mathematical Modelling}
\subsection{Co-ordinate systems and Reference Frames}

The spacecraft moves in an inertial reference frame $N$ with associated cartesian coordinate system $OXYZ$ where $\mathbf{n_1}$, $\mathbf{n_2}$ and $\mathbf{n_3}$ are unit vectors along the $X$, $Y$ and $Z$ axes respectively. This coordinate system is centred at the center of the earth, $\mathbf{n_3}$ is normal to the equatorial plane and parallel to the earth rotational axis, pointing towards north pole, $\mathbf{n_1}$ points towards the vernal equinox, and $\mathbf{n_2}$ completes the right-handed orthonormal frame.

A second reference frame $A$ is attached to the spacecraft rigid body with coordinate system $Cxyz$ where $C$ is the mass centre of the rigid body and $\mathbf{a_1}$, $\mathbf{a_2}$ and $\mathbf{a_3}$ are unit vectors along the $x$, $y$ and $z$ axes respectively.

A third reference frame, named orbital frame, and here denoted $O$, is spanned by the unit vectors $\mathbf{o_1}$, $\mathbf{o_2}$ and $\mathbf{o_3}$, and centred at the satellite's center of mass. The vector $\mathbf{o_3}$ constantly points to the Earth's center, whereas $\mathbf{o_2}$ aligns with the orbit angular momentum. $\mathbf{o_2}$ is the cross product of $\mathbf{o_3}$ and $\mathbf{o_1}$, completing the three axis orthonormal frame.

\subsection{Satellite Dynamics}

EIRSAT-1 is modelled as a single rigid body with 6-DOF (degrees of freedom).
The satellite has mass $m$ and inertia tensor $I$ in the body fixed coordinate system

\subsection{Rotational Kinematics}

The attitude of the spacecraft is described by quaternions.

\subsubsection{Alternative representations of attitude and conversions}
The orientation of the spacecraft may also be described using a Direction Cosine Matrix.

The orientation of the spacecraft may also be described using Euler angles.
Starting with the body-fixed $oxyz$ axes aligned with the $OXYZ$ axes, the body undergoes a sequence of rotations by angles $\theta_3$, $\theta_2$ and $\theta_1$ about the body-fixed $z$ (yaw), $y$ (pitch) and $x$ (roll) axes respectively to reach its final orientation.

The DCM (Direction Cosine Matrix) is then described by equation \ref{eq:dcm}.
\begin{equation} \label{eq:dcm}
C = 
\end{equation}

\subsection{Orbit Propagation}

\subsection{Environment}
\subsubsection{Magnetic Field}
\subsubsection{Gravitational Field}

\subsection{Sensor Models}

\subsection{Actuator Models}

\section{Controller Design}

\section{Simulink Model Structure}
\subsection{Dynamics block}
	\subsubsection{Spacecraft Dynamics} %All
	\subsubsection{Orbital Dynamics} % Daire
\subsection{Environment block} % Victor
\subsection{Disturbance block} % Victor
\subsection{Sensor block} % Victor
\subsection{Actuator block} % Daire
\subsection{Controller block} % All
\subsection{Reference block} % Joe
\subsection{Mode Select block} % Joe
\subsection{Running Simulations}
Model Setup File

Simulation Setup File

\section{Results}
