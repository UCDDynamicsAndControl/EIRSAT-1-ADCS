% !TEX root = EIRSAT_ADCS_Model.tex

\section{Mathematical Modelling}
\subsection{Co-ordinate systems and Reference Frames}

The spacecraft moves in an inertial reference frame $N$ with associated cartesian coordinate system $OXYZ$ where $\mathbf{n_1}$, $\mathbf{n_2}$ and $\mathbf{n_3}$ are unit vectors along the $X$, $Y$ and $Z$ axes respectively. This coordinate system is centred at the center of the earth, $\mathbf{n_3}$ is normal to the equatorial plane and parallel to the earth rotational axis, pointing towards north pole, $\mathbf{n_1}$ points towards the vernal equinox, and $\mathbf{n_2}$ completes the right-handed orthonormal frame.

A second reference frame $A$ is attached to the spacecraft rigid body with coordinate system $Cxyz$ where $C$ is the mass centre of the rigid body and $\mathbf{a_1}$, $\mathbf{a_2}$ and $\mathbf{a_3}$ are unit vectors along the $x$, $y$ and $z$ axes respectively.

A third reference frame, named orbital frame, and here denoted $O$, is spanned by the unit vectors $\mathbf{o_1}$, $\mathbf{o_2}$ and $\mathbf{o_3}$, and centred at the satellite's center of mass. The vector $\mathbf{o_3}$ constantly points to the Earth's center, whereas $\mathbf{o_2}$ aligns with the orbit angular momentum. $\mathbf{o_2}$ is the cross product of $\mathbf{o_3}$ and $\mathbf{o_1}$, completing the three axis orthonormal frame.

\subsection{Satellite Dynamics}

EIRSAT-1 is modelled as a single rigid body with 6-DOF (degrees of freedom).
The satellite has mass $m$ and inertia tensor $I$ in the body fixed coordinate system

\subsection{Rotational Kinematics}

The attitude of the spacecraft is described by quaternions.

\subsubsection{Alternative representations of attitude and conversions}
The orientation of the spacecraft may also be described using a Direction Cosine Matrix.

The orientation of the spacecraft may also be described using Euler angles.
Starting with the body-fixed $oxyz$ axes aligned with the $OXYZ$ axes, the body undergoes a sequence of rotations by angles $\theta_3$, $\theta_2$ and $\theta_1$ about the body-fixed $z$ (yaw), $y$ (pitch) and $x$ (roll) axes respectively to reach its final orientation.

The DCM (Direction Cosine Matrix) is then described by equation \ref{eq:dcm}.
\begin{equation} \label{eq:dcm}
C = 
\end{equation}

\subsection{Orbit Propagation}

\subsection{Environment}
	% !TeX root = sections.tex

{\bf Spherical Harmonics}

The set of shperical harmonics is an orthonormal set of functions defined in the unit sphere. They are given as the angular part of the solution to the Laplace Equation in spherical coordinates. Analogously  to how sines and cosines are used in harmonic analysis to represent periodic functions through Fourier Series, Spherical Harmonics may be used to define functions in the surface of a sphere.

{\bf Spherical Coordinates}

It is important to identify the coordinates to be used, as, for instance, spherical coordinates have often two different interpretations: one, which is mainly used by mathematicians, uses the latitude, being the angle between the position vector and the equatorial plane, as one of the angular coordinates; while the other, broadly employed by physicists, uses the co-latitude, defined as the angle between the position vector and the $z$ axis, thus being the complementary of the latitude. 

Here, the spherical coordinates are taken to be
\begin{itemize} 
\item[] $r$, the radial distance from Earth's center
\item[] $\phi$, the East longitude, bounded between $-\pi$ and $\pi$
\item[] $\theta$, the co-latitude, defined to be $\theta = \frac{\pi}{2} - \theta'$, with $\theta'$ the latitude. Bounded between $0$ and $\pi$
\end{itemize}


The magnetic field is given by the negative gradient of the potential $V$ defined for $r \geq R_e$, and given by the spherical harmonic approximation

\begin{equation} \label{eq:igrf_potential}
V = r \sum_{n=1}^{L} \left(\dfrac{R_e}{r}\right)^{n+1} \sum_{m=0}^{n} \left(g_n^m cos(m\phi) + h_n^m sin(m\phi)\right) P_n^m(cos\theta)
\end{equation}

That is
\begin{equation}
{\bm \beta} = -\nabla V
\end{equation}

Where $g_n^m$ and $h_n^m$ are the set of IGRF gaussian coefficients, published and revised every five years by the participating members of the IAGA (International Association
of Geomagnetism and Aeronomy). The $12^{th}$ generation is here used, as the last revision by the time of this work. These coefficient also include the secular variation (SV), which bestow the model on the proper time dependence. Keping track of the change in the coefficients in $nT$ per year. The coefficients  for some epoch year are referred to as IGRF. When real data about the geomagnetic field becomes available so that some adjustements can be made, the model becomes definitive and changes its name to DGRF (Deffinitive geomagnetic reference field).

$P_n^m(cos \theta)$ are the Schmidt normalized associated Legendre polynomials of degree n and order m

One can find very effective, recursive ways of computing the Schmidt quasi-normalization factors, the Associated Legendre polynomials and its derivatives, as can be found on [J.Davis - Mathematical Modeling of Earth’s Magnetic Field] %\cite{...}

The three componentens of the magnetic field in $r$, $\phi$ and $\theta$ directions can be found in terms of the partial derivatives of \ref{eq:igrf_potential}

\begin{equation}
{\bm \beta}_r = -\dfrac{\partial V}{\partial r} = \sum_{n=1}^{N} (n+1) \left(\dfrac{R_e}{r}\right)^{n+2} \sum_{m=0}^{n} \left(g_n^m cos(m\phi) + h_n^m sin(m\phi)\right) P_n^m(cos\theta)
\end{equation}

\begin{equation}
{\bm \beta}_{\theta} = -\dfrac{1}{r} \dfrac{\partial V}{\partial \theta} = 
-\sum_{n=1}^{L} \left(\dfrac{R_e}{r}\right)^{n+2} \sum_{m=0}^{n} \left(g_n^m cos(m\phi) + h_n^m sin(m\phi)\right) \dfrac{\partial P_n^m(cos\theta)}{\partial \theta}
\end{equation}

\begin{equation}
{\bm \beta}_{\theta} = -\dfrac{1}{r sin\theta} \dfrac{\partial V}{\partial \phi} = 
\dfrac{1}{sin\phi}\sum_{n=1}^{L} \left(\dfrac{R_e}{r}\right)^{n+2} \sum_{m=0}^{n} m\left(g_n^m sin(m\phi) - h_n^m cos(m\phi)\right) P_n^m(cos\theta)
\end{equation}

{\bf Legendre Polynomials}

The set of regular Legendre Polynomials is given by the solutions $P_n(\nu)$ to the equation
\begin{equation}
\dfrac{1}{\sqrt{1-2\nu x + x^2}} = \sum_{n=0}^\infty x^2 P_n(\nu)
\end{equation}
Which gives
\begin{equation}
P_n(\nu) = \dfrac{1}{n! 2^n} \dfrac{d^n}{d\nu^n} (\nu^2 -1)^n
\end{equation}

{\bf Associated Legendre Polynomials}

The associated Legendre polynomials $P_n^m(\nu)$ of degree n and order m relate to $P_n(\nu)$ by
\begin{equation}
P_{n,m}(\nu) = (1-\nu^2)^{\frac{1}{2m}} \dfrac{d^m}{d\nu^m}(P_n(\nu))
\end{equation}
These polynomias are not normalized in any way. 

{\bf Gaussian normalization and Schmidt quasi-normalization}

To compute the geomagnetic field, the Schmidt quasi-normalization is used, and is given by
\begin{equation}
P_n^m(\nu) = \sqrt{\dfrac{2(n-m)!}{(n+m)!}} P_{n,m}(\nu) 
\end{equation}

The Gaussian normalizalized Legendre polynomials, whose interest lies in the existence of a recursive formula for their effective computation are given by
\begin{equation}
P^{n,m}(\nu) = \dfrac{2^n! (n-m)!}{(2n)!} P_{n,m}(\nu)
\end{equation}

One can relate the Gaussian normalization and the Schmidt quasi-normalization by using
\begin{equation}
P_n^m(\nu) = \Gamma_{n,m} P^{n,m}(\nu)
\end{equation}
Where
\begin{equation}
\begin{aligned}
\Gamma_{n,m} &= \sqrt{\dfrac{(n-m)!}{(n+m)!}}\dfrac{(2n-1)!!}{(n-m)!}, &m = 0\\
\Gamma_{n,m} &= \sqrt{\dfrac{2(n-m)!}{(n+m)!}}\dfrac{(2n-1)!!}{(n-m)!}, &m \neq 0
\end{aligned}
\end{equation}

It is more convinient to modify the coefficients, as
\begin{equation}
\begin{aligned}
g_n^m &= \Gamma_{n,m}\ g_{n,m}\\ h_n^m &= \Gamma_{n,m}\ h_{n,m}
\end{aligned}
\end{equation}


\footnote{Note the use of notation here to distinguish between Regular, Associated, and Schmidt quasi-normalized Legendre Polynomials}


\subsubsection{Magnetic Field}
\subsubsection{Gravitational Field}

\subsection{Sensor Models}

\subsection{Actuator Models}

\section{Controller Design}

