% !TEX root = EIRSAT_ADCS_Model.tex

\section{Mathematical Modelling}
\subsection{Co-ordinate systems and Reference Frames}

The spacecraft moves in an inertial reference frame $N$ with associated Cartesian coordinate system $O_{XYZ}$ where $\left\{ \mathbf{n_1}, \mathbf{n_2}, \mathbf{n_3} \right\}$ is a set of unit vectors along the $X$, $Y$ and $Z$ axes respectively. This coordinate system is centred at the centre of the Earth, $\mathbf{n_3}$ is normal to the equatorial plane (hence parallel to the earth rotational axis), and points towards north pole, $\mathbf{n_1}$ points towards the vernal equinox, and $\mathbf{n_2}$ completes the right-handed orthonormal frame.\\
Recall that $N$ is considered an inertial reference, neglecting the rotation of the Earth around the sun, and that of the sun around bigger objects.

A second reference frame $A$, is attached to the spacecraft rigid body with coordinate system $C_{xyz}$ where $C$ is the mass centre of the rigid body and $ \left\{\mathbf{a_1}, \mathbf{a_2}, \mathbf{a_3} \right\}$ are unit vectors along the $x$, $y$ and $z$ axes respectively. It will be assumed that the three axes of the body fixed frame are aligned with the principal axes of inertia of the satellite. 

A third reference frame, named orbital frame, and here denoted $O$, is spanned by the unit vectors $\left\{\mathbf{o_1}, \mathbf{o_2}, \mathbf{o_3} \right\}$, and centred at the satellite's centre of mass. The vector $\mathbf{o_3}$ constantly points to the Earth's centre, whereas $\mathbf{o_2}$ aligns with the orbit angular momentum. $\mathbf{o_2}$ is the cross product of $\mathbf{o_3}$ and $\mathbf{o_1}$, completing the three axis orthonormal frame.

\subsection{Satellite Dynamics}

EIRSAT-1 is modelled as a single rigid body with six degrees of freedom: three rotational and three translational.\\
The satellite has mass $m$ and inertia tensor $\mathbf{I}$ in the body fixed coordinate system $O$. From the assumption that $O$ is aligned with the principal axes of inertia of the satellite, it follows that is a $3\times 3$ diagonal matrix.

\subsection{Rotational Kinematics}

The attitude of the spacecraft is described using unit quaternions. These avoid singularities that may arise when other approaches such as DCM (Direction Cosine Matrix) or Euler Angle - Euler Axis are used. Due to the lack of singularities, unit quaternions are widely employed through the literature for computational purposes in ADCS design and simulations. See, for instance ([???]).\\

\subsubsection{Alternative representations of attitude and conversions}

{|bf Direction Cosine Matrix}
Rotations in three-dimensional Euclidean space may be described in a number of ways. For instance, let $\mathbf{r}^X$ be a vector expressed as a linear combination of the unit vectors that span the right - handed orthonormal frame $X = \left\{\mathbf{e}_1, \mathbf{e}_2, \mathbf{e}_3 \right\}$. That is
\begin{equation}
\mathbf{r} = \sum_{i=1}^{3}r_i \mathbf{e_i}
\end{equation}

The same vector can be represented by a linear combination of unit vectors spanning an alternative othonormal frame {$X' = \left\{\mathbf{e}'_1, \mathbf{e}'_2, \mathbf{e}'_3 \right\}$}

\begin{equation}
\mathbf{r} = \sum_{i=1}^{3}r'_i \mathbf{e'_i}
\end{equation}

In vector form:

\begin{equation}
	\mathbf{r} = \left(\begin{array} {c c c}\bfe_1 & \bfe_2 & \bfe_3 \end{array} \right) \left( \begin{array}{c} r_1\\ r_2\\\ r_3 \end{array}\right) =
	\left( \begin{array} {c c c}\bfe'_1 & \bfe'_2 & \bfe'_3 \end{array} \right) \left( \begin{array}{c} r'_1\\ r'_2\\\ r'_3 \end{array}\right) =
\end{equation}

Each one of the components $r'_i$ is the projection (ie. vector product) of the vector $\mathbf{r}$ into $\bfe'_i$. That is $r'_i = \mathbf{r} \cdot \bfe'_i =  \sum_{j=1}^{3} (r_j \bfe_j)\cdot \bfe'_i = \sum_{j=1}^{3} (\bfe'_i \cdot \bfe_j) r_j$. In matrix form:

\begin{equation}
\left( \begin{array}{c} r'_1 \\ r'_2 \\ r'_3 \end{array} \right) = 
\left( \begin{array}{c c c}  
\bfe'_1\cdot\bfe_1 & \bfe'_1\cdot\bfe_2 & \bfe'_1\cdot\bfe_3 \\
\bfe'_2\cdot\bfe_1 & \bfe'_2\cdot\bfe_2 & \bfe'_2\cdot\bfe_3 \\
\bfe'_3\cdot\bfe_1 & \bfe'_3\cdot\bfe_2 & \bfe'_3\cdot\bfe_3 \end{array} \right) 
\left( \begin{array}{c} r_1\\ r_2 \\ r_3 \end{array} \right)
\end{equation}

There exists a matrix $\mathbf{A}_X^{X'} \in\ SO(3)$ that allows the transformation to the frame {$X' = \left\{\mathbf{e}'_1, \mathbf{e}'_2, \mathbf{e}'_3 \right\}$}. That is to say

\begin{align}
\mathbf{r}^{X'} &= \mathbf{A}_X^{X'} \mathbf{r}^{X}\\
\mathbf{r}^{X'} &= \sum_{i=1}^{3} a_{ij} r_j^X \mathbf{e}'_i
\end{align}
With
\begin{equation} \label{eq:DCM}
\mathbf{A}_X^{X'} =
\left( \begin{array}{c c c}  
\bfe'_1\cdot\bfe_1 & \bfe'_1\cdot\bfe_2 & \bfe'_1\cdot\bfe_3 \\
\bfe'_2\cdot\bfe_1 & \bfe'_2\cdot\bfe_2 & \bfe'_2\cdot\bfe_3 \\
\bfe'_3\cdot\bfe_1 & \bfe'_3\cdot\bfe_2 & \bfe'_3\cdot\bfe_3 \end{array} \right) 
\end{equation}
This implies 
\begin{equation}
\left( \begin{array}{c} \bfe'_1 \\ \bfe'_2 \\ \bfe'_3 \end{array} \right) = \mathbf{A}_X^{X'} \left( \begin{array}{c} \bfe_1 \\ \bfe_2 \\ \bfe_3 \end{array} \right)
\end{equation}

 $SO(3)$ is the group of rotations in three-dimensional euclidean space. If rotations are described using an orthonormal basis in $\mathbb{R}^3$, then all matrices in $SO(3)$ are $3\times 3$ orthogonal matrices: real matrices whose transpose equals its left and right inverse.\\

The same matrix $\mathbf{A}_X^{X'}$ that produces the change of coordinate system from $X$ to $X'$ when applied to a vector expressed in terms of $X$, may be used to rotate a vector $\mathbf{r}$, forcing it to perform the inverse rotation that $X'$ did with respect to $X$, to obtain antoher vector $\mathbf{u}$·. Intuitively, if one can place an observer fixed to $\mathbf{r}$ while such a rotation takes place, it would \textit{see} the reference frame $X$ transform to $X'$. Recall that when the rotation matrix is used for the latter purpose, no frame transformation takes place, and a single coordinate frame is used to represent each vector.\\

The matrix $\mathbf{A}_X^{X'}$ is named the \textbf{Direction Cosine Matrix} as it is formed by the dot products $\mathbf{A}_X^{X'} = \left\lbrace a_{ij} \right\rbrace = \left\lbrace \mathbf{e}'_i \cdot \mathbf{e}_j \right\rbrace $ which are, in fact, the cosines of the angles that the unit vectors $\mathbf{e}'_i$ form with respect to $\mathbf{e}_j$. (see figure \ref{fig:frameCosines}).\\
	
\begin{figure}
	\centering
	%\includegraphics[]{}
	\caption{Direction Cosines}
	\label{fig:frameCosines}
\end{figure}

{\bf Euler's Theorem. Euler Axis and Principal Angle}

Euler's Theorem ([...]) states that any rotation in 3D space representing an arbitrary transformation between a pair of coordinate frames is uniquely determined by an axis about which a rotation of an angle $\phi$ takes place. The convention is to take $\phi$ positive if the rotation is counter-clockwise, and negative if clockwise. $\phi$ is called the \textit{Principal Angle}. The \textit{Euler  Axis} is designated by a unit vector $\mathbf{e}$ \\

Euler Axis and Principal Angle are closely related to the eigenvalues and eigenvetors of the Direction Cosine Matrix (\ref{eq:DCM}) associated to the transformation. No further discussions on Euler Axis and Principal Angle are required for the complete understanding of this work, therefore we refer to \cite{tewari2007} for the interested reader.\\

{\bf Euler Angles}

The orientation of the spacecraft may also be described using Euler angles.
Starting with the body-fixed $oxyz$ axes aligned with the $OXYZ$ axes, the body undergoes a sequence of rotations by angles $\theta_3$, $\theta_2$ and $\theta_1$ about the body-fixed $z$ (yaw), $y$ (pitch) and $x$ (roll) axes respectively to reach its final orientation, as depicted in figure \ref{fig:EulerAngles}.

\begin{figure}
	\centering
	%\includegraphics[]{}
	\caption{Euler Angles. $\theta_3$, $\theta_2$, $\theta_3$  Rotation sequence}
	\label{fig:EulerAngles}
\end{figure}

The DCM (Direction Cosine Matrix) is then described by equation (\ref{eq:DCMfromEulAng}). See \cite{tewari2007}, or any other work in rigid body dynamics for a derivation.
 
\begin{equation} \label{eq:DCMfromEulAng}
\newcommand{\ct}{cos\theta_}
\newcommand{\st}{sin\theta_}
A_{X}^{X'} = \left( \begin{array}{c c c}
\ct2\ct3 & \ct2\st3 & -\st2 \\
\st1\st2\ct3 - \ct1\st3 & \st1\st2\st3 + \ct1\ct3 & \st1\ct2 \\
\ct1\st2\ct3 + \st1\st3 & \ct1\st2\st3 - \st1\ct3 & \ct1\ct2
\end{array}\right)
\end{equation}

Among all the alternatives, Euler Angles are the most intuitive tool for representing rotations in three dimensions, as they have a strong \textit{visual} interpretation. Following the approach used in similar works (see, for instance [???], [???]), Euler Angles have been chosen for attitude representation. Nonetheless, quaternions are used for computational purposes, and appropriate conversions are performed prior to representation of the results.

{\bf Quaternions}

The Euler Symmetric Parameters, also named Quaternions, are sets of four scalar quantities $\mathbf{q} = \left( \eta,  \epsilon_1, \epsilon_2, \epsilon_3 \right)^T$, comprising themselves a scalar part $\eta$ and a vector part $\mathbf{\epsilon} = \left(\epsilon_1, \epsilon_2, \epsilon_3\right)^T$.\\

The definition of a quaternion associated with a transformation in 3D is given by the Euler Axis $\mathbf{e}$ and Principal Angle $\phi$ that uniquely define such transformation:
\begin{equation}
\begin{aligned}
\mathbf{\epsilon}  & = \mathbf{e} \ sin(\phi/2)  \\
\eta & = cos(\phi/2)
\end{aligned}
\end{equation}
From the definition, it follows that the four scalar parts which form the quaternion are mutually dependent: $ \eta^2 + \epsilon_1^2 + \epsilon_2^2 + \epsilon_3^2 = 1$\\

Both representations Euler Axis/Principal Angle and Quaternions are singularity-free. However, as was already pointed out, using Quaternions offer advantages that are mainly related to computational issues. For instance, the rotation matrix is directly derived from the vector and scalar part of the quaternion without the need for trigonometric calculations, which are generally computationally expensive.
\begin{equation}
A_X^{X'} = (\eta^2 - \pmb{\epsilon}^T \pmb{\epsilon}) I +
2\pmb{\epsilon}\pmb{\epsilon}^T - 2\eta Sk(\pmb{\epsilon})
\end{equation}
Where $Skw(\cdot): \mathbb{R}^3 \longrightarrow \mathbb{R}^{3\times 3}$ is an operator related to the vector product 
\begin{equation}
Sk(\mathbf{u}) = 
\left(\begin{array}{c c c}
0 & -u_3 & u_2 \\
u_3 & 0 & -u_1 \\
-u_2 & u_1 & 0
\end{array} \right)
\end{equation}
Such that $\mathbf{u}\times\mathbf{v} = Sk(\mathbf{u})\mathbf{v}$.\\

Moreover, the quaternion is also easily expressed in terms of the elements $a_{uj}$ of the direction cosine matrix $A_X^{X'}$.
	
\begin{equation}
\begin{aligned}
\eta & = \frac{\sqrt{1+tr\left(A_X^{X'}\right)}}{2}\\
	\epsilon_1 & = \frac{a_{23} - a_{32}}{4\eta}\\
	\epsilon_2 & = \frac{a_{31} - a_{13}}{4\eta}\\
	\epsilon_3 & = \frac{a_{12} - a_{21}}{4\eta}
	\end{aligned}
\end{equation}


Every attitude representation implies only three independent parameters

\subsection{Orbit Propagation}

\subsection{Environment}
% !TeX root = sections.tex

{\bf Spherical Harmonics}

The set of shperical harmonics is an orthonormal set of functions defined in the unit sphere. They are given as the angular part of the solution to the Laplace Equation in spherical coordinates. Analogously  to how sines and cosines are used in harmonic analysis to represent periodic functions through Fourier Series, Spherical Harmonics may be used to define functions in the surface of a sphere.

{\bf Spherical Coordinates}

It is important to identify the coordinates to be used, as, for instance, spherical coordinates have often two different interpretations: one, which is mainly used by mathematicians, uses the latitude, being the angle between the position vector and the equatorial plane, as one of the angular coordinates; while the other, broadly employed by physicists, uses the co-latitude, defined as the angle between the position vector and the $z$ axis, thus being the complementary of the latitude. 

Here, the spherical coordinates are taken to be
\begin{itemize} 
\item[] $r$, the radial distance from Earth's center
\item[] $\phi$, the East longitude, bounded between $-\pi$ and $\pi$
\item[] $\theta$, the co-latitude, defined to be $\theta = \frac{\pi}{2} - \theta'$, with $\theta'$ the latitude. Bounded between $0$ and $\pi$
\end{itemize}


The magnetic field is given by the negative gradient of the potential $V$ defined for $r \geq R_e$, and given by the spherical harmonic approximation

\begin{equation} \label{eq:igrf_potential}
V = r \sum_{n=1}^{L} \left(\dfrac{R_e}{r}\right)^{n+1} \sum_{m=0}^{n} \left(g_n^m cos(m\phi) + h_n^m sin(m\phi)\right) P_n^m(cos\theta)
\end{equation}

That is
\begin{equation}
{\bm \beta} = -\nabla V
\end{equation}

Where $g_n^m$ and $h_n^m$ are the set of IGRF gaussian coefficients, published and revised every five years by the participating members of the IAGA (International Association
of Geomagnetism and Aeronomy). The $12^{th}$ generation is here used, as the last revision by the time of this work. These coefficient also include the secular variation (SV), which bestow the model on the proper time dependence. Keping track of the change in the coefficients in $nT$ per year. The coefficients  for some epoch year are referred to as IGRF. When real data about the geomagnetic field becomes available so that some adjustements can be made, the model becomes definitive and changes its name to DGRF (Deffinitive geomagnetic reference field).

$P_n^m(cos \theta)$ are the Schmidt normalized associated Legendre polynomials of degree n and order m

One can find very effective, recursive ways of computing the Schmidt quasi-normalization factors, the Associated Legendre polynomials and its derivatives, as can be found on [J.Davis - Mathematical Modeling of Earth’s Magnetic Field] %\cite{...}

The three componentens of the magnetic field in $r$, $\phi$ and $\theta$ directions can be found in terms of the partial derivatives of \ref{eq:igrf_potential}

\begin{equation}
{\bm \beta}_r = -\dfrac{\partial V}{\partial r} = \sum_{n=1}^{N} (n+1) \left(\dfrac{R_e}{r}\right)^{n+2} \sum_{m=0}^{n} \left(g_n^m cos(m\phi) + h_n^m sin(m\phi)\right) P_n^m(cos\theta)
\end{equation}

\begin{equation}
{\bm \beta}_{\theta} = -\dfrac{1}{r} \dfrac{\partial V}{\partial \theta} = 
-\sum_{n=1}^{L} \left(\dfrac{R_e}{r}\right)^{n+2} \sum_{m=0}^{n} \left(g_n^m cos(m\phi) + h_n^m sin(m\phi)\right) \dfrac{\partial P_n^m(cos\theta)}{\partial \theta}
\end{equation}

\begin{equation}
{\bm \beta}_{\theta} = -\dfrac{1}{r sin\theta} \dfrac{\partial V}{\partial \phi} = 
\dfrac{1}{sin\phi}\sum_{n=1}^{L} \left(\dfrac{R_e}{r}\right)^{n+2} \sum_{m=0}^{n} m\left(g_n^m sin(m\phi) - h_n^m cos(m\phi)\right) P_n^m(cos\theta)
\end{equation}

{\bf Legendre Polynomials}

The set of regular Legendre Polynomials is given by the solutions $P_n(\nu)$ to the equation
\begin{equation}
\dfrac{1}{\sqrt{1-2\nu x + x^2}} = \sum_{n=0}^\infty x^2 P_n(\nu)
\end{equation}
Which gives
\begin{equation}
P_n(\nu) = \dfrac{1}{n! 2^n} \dfrac{d^n}{d\nu^n} (\nu^2 -1)^n
\end{equation}

{\bf Associated Legendre Polynomials}

The associated Legendre polynomials $P_n^m(\nu)$ of degree n and order m relate to $P_n(\nu)$ by
\begin{equation}
P_{n,m}(\nu) = (1-\nu^2)^{\frac{1}{2m}} \dfrac{d^m}{d\nu^m}(P_n(\nu))
\end{equation}
These polynomias are not normalized in any way. 

{\bf Gaussian normalization and Schmidt quasi-normalization}

To compute the geomagnetic field, the Schmidt quasi-normalization is used, and is given by
\begin{equation}
P_n^m(\nu) = \sqrt{\dfrac{2(n-m)!}{(n+m)!}} P_{n,m}(\nu) 
\end{equation}

The Gaussian normalizalized Legendre polynomials, whose interest lies in the existence of a recursive formula for their effective computation are given by
\begin{equation}
P^{n,m}(\nu) = \dfrac{2^n! (n-m)!}{(2n)!} P_{n,m}(\nu)
\end{equation}

One can relate the Gaussian normalization and the Schmidt quasi-normalization by using
\begin{equation}
P_n^m(\nu) = \Gamma_{n,m} P^{n,m}(\nu)
\end{equation}
Where
\begin{equation}
\begin{aligned}
\Gamma_{n,m} &= \sqrt{\dfrac{(n-m)!}{(n+m)!}}\dfrac{(2n-1)!!}{(n-m)!}, &m = 0\\
\Gamma_{n,m} &= \sqrt{\dfrac{2(n-m)!}{(n+m)!}}\dfrac{(2n-1)!!}{(n-m)!}, &m \neq 0
\end{aligned}
\end{equation}

It is more convinient to modify the coefficients, as
\begin{equation}
\begin{aligned}
g_n^m &= \Gamma_{n,m}\ g_{n,m}\\ h_n^m &= \Gamma_{n,m}\ h_{n,m}
\end{aligned}
\end{equation}


\footnote{Note the use of notation here to distinguish between Regular, Associated, and Schmidt quasi-normalized Legendre Polynomials}


\subsubsection{Magnetic Field}
\subsubsection{Gravitational Field}

\subsection{Sensor Models}

\subsection{Actuator Models}

\section{Controller Design}
